 \documentclass[parskip=half]{scrartcl}

\usepackage[utf8]{inputenc}
\usepackage[ngerman]{babel}
\usepackage{amsmath}
\usepackage{amsfonts}
\usepackage{amssymb}
\usepackage{color}
\usepackage{hyperref}
\usepackage{graphicx}
\usepackage{epstopdf}
\usepackage{listings}
\usepackage{fancyvrb}
\definecolor{lnkcol}{rgb}{0.0,0.0,0.6}
\hypersetup{
	pdfstartview=FitH,
	colorlinks=true,
	urlcolor=lnkcol,
	citecolor=lnkcol,
	linkcolor=lnkcol,
	filecolor=lnkcol
}

\title{PiratenID: Handbuch für Webentwickler}
\author{Jan Schejbal}
\date{}

\bibliographystyle{unsrt}

\setcounter{secnumdepth}{2}

\begin{document}
\maketitle

\begin{abstract}\noindent
Dieses Dokument enthält eine Anleitung für Webseitenbetreiber, die das PiratenID-System in ihre Website einbinden möchten.
Für reguläre Nutzer existiert eine separate Anleitung\footnote{TODO}.
Technische Details zum ID-System finden sich in der technischen Dokumentation\footnote{TODO}.
\end{abstract}


\section{Einführung}
Das PiratenID-System soll es Mitgliedern der Piratenpartei ermöglichen, ihren Mitgliedschaftsstatus gegenüber von Dritten betriebenen Webseiten nachzuweisen,
ohne dass Zugriff auf die Mitgliederdatenbank nötig ist und ohne mehr Daten als nötig offenlegen zu müssen.

Somit ermöglicht PiratenID es Websitebetreibern unter anderem, Dienste nur für Piraten anzubieten und sicherzustellen, dass pro Pirat nur ein Account angelegt wird.
Für die Nutzung von PiratenID ist keine Anmeldung oder Genehmigung notwendig.

Das PiratenID-System basiert auf OpenID 2.0 \footnote{\url{http://openid.net/specs/openid-authentication-2_0.html}} (mit gewissen Einschränkungen)
mit Attribute Exchange 1.0 \footnote{\url{http://openid.net/specs/openid-attribute-exchange-1_0.html}}.

Für PHP wird Code bereitgestellt, der neben der Anbindung an PiratenID auch Teile des Session-Handlings übernehmen kann.
Alternativ kann das PiratenID-System mit kompatiblen OpenID 2.0-Libraries genutzt werden.
Welche Punkte dabei zu beachten sind, wird in den jeweiligen Abschnitten erläutert.

Zusätzlich zu dieser Anleitung könnte die technische Dokumentation\footnote{TODO},
in welcher die technischen Hintergründe ausführlicher erklärt werden, interessant sein.

\newpage
\section{Was bietet PiratenID?}
Websites und Webanwendungen können PiratenID nutzen, um Benutzer zu identifizieren und ihren Mitgliedsstatus in der Piratenpartei zu prüfen.
Hierzu erstellt die Website eine Identifizierunganfrage und leitet den Nutzer auf die Seite des PiratenID-Systems um.
Dort prüft der Nutzer, welche Daten abgefragt werden, meldet sich mit mit seiner PiratenID-Benutzerkennung an und gibt die Daten somit frei.
Der Nutzer wird auf die anfragende Website zurückgeleitet und die Website erhält die freigegebenen Daten.
Personenbezogene Daten wie Name und Mitgliedsnummer können \textbf{nicht} abgefragt werden.

Neben dem Mitgliedschaftsstatus (inkl. Untergliederungen) kann optional auch ein Pseudonym abgefragt werden.
Das Pseudonym ermöglicht es einer Webanwendung, einen Benutzer wiederzuerkennen.
Es handelt sich hierbei nicht um einen vom Nutzer gewählten Namen, sondern um einen berechneten Wert.
Die Pseudonyme sind für jede Website (genauer: für jedes OpenID-Realm) anders.
Pro Website (Realm) hat jeder Nutzer aber genau ein unveränderliches Pseudonym.

Eine anonyme Anmeldung ohne Pseudonymabfrage ist z. B. sinnvoll, um den Zugang zu einem Downloadbereich auf Piraten zu beschränken.
Sollen Nutzer wiedererkannt werden (z. B. um eine Mehrfachteilnahme an Umfragen zu verhindern), muss das Pseudonym abgefragt werden.

\subsection{Nutzungsarten}
Für klassische Webanwendungen, die Benutzerkonten benötigen, kann das PiratenID-System auf zwei verschiedene Arten genutzt werden.
Der angebotene PHP-Client unterstützt beide.

\subsubsection{Login via PiratenID}
Die einfachste Möglichkeit ist, das Login komplett von PiratenID übernehmen zu lassen.
Wird die Anwendung in PHP entwickelt, kann diese Nutzungsart mittels des angebotenen PHP-Clients mit wenigen Zeilen Code realisiert werden.
Alternativ kann OpenID verwendet werden (ggf. wird das von manchen Anwendungen nativ unterstützt).

Möchte sich ein Nutzer einloggen, wird eine PiratenID-Authentifikation durchgeführt.
Der Pirat loggt sich nur bei PiratenID ein, die Webanwendung bekommt bei Erfolg das Pseudonym mitgeteilt und kann dieses wie eine Benutzer-ID verwenden.
Die Webanwendung verwaltet selbst keine Passwörter, was bei Eigenentwicklungen viel Arbeit (und sicherheitskritischen Code) spart.
Soll ein Nutzer gesperrt werden, kann dies über das Pseudonym geschehen.

Bei fertigen Anwendungen mit eigener Nutzerverwaltung kann diese Möglichkeit meist dennoch genutzt werden, indem die vorhandene Loginfunktion entsprechend angepasst wird:
Bei der Rückkehr von einer erfolgreichen PiratenID-Anfrage wird der Nutzer eingeloggt, zu dem das Pseudonym passt
(dieses/ein Hash davon ist der Benutzername oder in einer eigenen Spalte der Benutzertabelle gespeichert).
Sollte kein solcher Nutzer existieren, wird er angelegt, ggf. nötige Daten werden abgefragt (z. B. anzuzeigender Nickname).

Neben der einfachen Implementierung hat dieses Verfahren den Vorteil, dass der Nutzer kein weiteres Kennwort braucht (Single-Sign-On).
Außerdem wird bei jedem Login geprüft, ob der Nutzer noch Pirat ist. Ist er z. B. ausgetreten, bekommt die Anwendung das spätestens beim nächsten Login mit.
Die Anwendung kann bei dieser Variante oft darauf verzichten, personenbezogene Daten wie eine E-Mail-Adresse zu sammeln.

Obwohl das System vor allem zur Verwendung durch Piraten gedacht ist, können auch Nichtpiraten einen PiratenID-Account anlegen.
Auch Anwendungen, die nicht nur Piraten offen stehen, können daher diese Anmeldevariante nutzen.

Ein Nachteil dieses Verfahrens ist die Abhängigkeit vom PiratenID-System (fällt dieses aus, ist kein Login mehr möglich).
Da das Login immer über das ID-System erfolgt, bekommt der ID-Server mit, wann sich welcher Nutzer am Dienst anmeldet.


\subsubsection{Eigenes Login mit Verifikation über PiratenID}
Alternativ kann die Webanwendung auch ihre eigene Benutzerverwaltung nutzen.
Dies kann bei bereits existierenden Anwendungen, einer bestehenden Nutzerbasis oder Anwendungen mit vielen externen Nutzern sinnvoller sein.

Möchte ein Pirat innerhalb der Anwendung als Pirat erkannt werden, loggt er sich zunächst in der Anmeldung ein.
Die Anwendung prüft die Piratenmitgliedschaft anschließend, indem sie (über den PHP-Client oder eine OpenID-Library) eine PiratenID-Authentifizierung durchführt.
Soll sichergestellt werden, dass jeder Pirat nur ein Benutzerkonto anlegen kann, wird zusätzlich das Pseudonym abgefragt und gespeichert.

Sollte der Nutzer aus der Partei austreten, wird die Anwendung darüber \textbf{nicht} benachrichtigt!
Daher muss die Überprüfung in regelmäßigen Abständen wiederholt werden.

Der Vorteil ist die Unabhängigkeit vom PiratenID-System, Nachteile sind der erhöhte Programmieraufwand bei Eigenentwicklungen,
separate Benutzerkonten und die verzögerte Erkennung von Parteiaustritten.

\subsection{Abfragbare Daten}
\label{sec:attribute}
Neben dem Pseudonym können zu jedem Nutzer die folgenden Attribute angefragt werden.
Der Nutzer muss die Übermittlung der Daten jeweils ausdrücklich bestätigen.

\begin{itemize}
	\item \texttt{mitgliedschaft-bund}
	\item \texttt{mitgliedschaft-land}
	\item \texttt{mitgliedschaft-bezirk}
	\item \texttt{mitgliedschaft-kreis}
	\item \texttt{mitgliedschaft-ort}
\end{itemize}

Ist ein Attribut in der Antwort leer, kann dies daran liegen, dass kein gültiges Piraten-Token eingetragen ist oder der Wert unbekannt ist.
Um absichtlich leere Attributfelder zu kennzeichnen (z. B. Pirat ist in keinem Ortsverband) werden drei Bindestriche (TODO WERT) verwendet.

Das Attribut "`\texttt{mitgliedschaft-bund}"' hat eine besondere Bedeutung:
Wird es \textbf{nicht} abgefragt, können sich nur Piraten einloggen (implizite Mitgliedschaftsprüfung).
Wird es abgefragt, können sich auch Nutzer einloggen, die keine Piraten sind - der Mitgliedschaftsstatus wird aber angegeben (explizite Mitgliedschaftsprüfung).
Ein Nutzer ist genau dann Mitglied, wenn in diesem Feld "`ja"' steht.
Andere Antworten (insbes. leeres Feld) bedeuten, dass der Nutzer kein Pirat ist (oder seine Mitgliedschaft nicht durch Eingabe des Tokens im ID-System eingetragen hat).


Das Pseudonym sollte als zusätzliche Sicherheitsmaßnahme möglichst verdeckt bleiben, also insbesondere dem Nutzer nicht angezeigt werden.
Sollte ein Angriff auf das ID-System gelingen, kann ein Angreifer die Identität des Nutzers nicht vortäuschen, wenn er sein Pseudonym nicht ermitteln kann.
(Technisch versierte Nutzer können das eigene Pseudonym durch Beobachtung des Loginvorgangs ermitteln.)
Am sinnvollsten ist es hierzu, das Pseudonym direkt nach Erhalt zu hashen (der bereitgestellte PHP-Client tut dies automatisch) und nur den Hash zu verwenden.


\newpage
\section{Verwendung über den PHP-Client}
Der bereitgestellte PiratenID-Client ist eine vollständige Eigenentwicklung.
Er ist in PHP geschrieben und ermöglicht Webanwendungen in der gleichen Sprache eine bequeme und sichere Nutzung des PiratenID-Systems.

Die Parameter des PiratenID-Systems sind fest in den Client eingebaut.
Der Client ist nicht für die Nutzung mit anderen OpenID-Servern gedacht oder geeignet.
(Es fehlen wichtige Teile des OpenID-Protokolls, welche für die Nutzung mit einem fest vorgegebenen Server jedoch nicht nötig sind.)

Der Client besteht aus einer PHP-Klasse mit statischen Funktionen und einigen statischen, öffentlichen Variablen.
Um ihn zu nutzen, werden (nach dem Einbinden der Datei) zunächst über die Variablen die gewünschten Parameter eingestellt.
Danach können die enthaltenen Funktionen benutzt werden.

Der Client kann das komplette Session-Handling übernehmen.
Hierzu müssen die betroffenen Funktionen (run bzw. initSession) aufgerufen werden, bevor die Header gesendet worden sind, d.h. vor allen Ausgaben.
Vorsicht: Ein Byte-order-mark, was manche Editoren am Anfang von UTF-8-Textdateien einfügen, kann in einer PHP-Datei dazu führen, dass die Header gesendet werden!
Wenn der Client das Session-Handling übernimmt, werden automatisch einige zusätzliche Sicherheitsmaßnahmen angewendet.
Falls es also möglich ist, sollte das Session-Handling immer dem Client überlassen werden.

Den einfachsten Einstieg in die Nutzung des Clients bietet das mitgelieferte Beispiel (\texttt{example.php}).

\subsection{Voraussetzungen}
Der Client benötigt PHP mit aktivierten Stream-Funktionen mitsamt HTTPS/SSL-Unterstützung.
Der Client wurde auf PHP 5.3.5 getestet.
Aus Sicherheitsgründen sollte stets eine aktuelle PHP-Version verwendet werden.

Die PHP-Einstellung \texttt{session.use\_only\_cookies} sollte zur Erhöhung der Sicherheit in der php.ini aktiviert werden, falls möglich.
Kann diese Einstellung nicht aktiviert werden, wird der Client die Session-ID bei jedem Aufruf neu erzeugen,
um session fixation-Angriffe zu verhindern.
Dies kann die Stabilität der Sessions beeinträchtigen.

\subsubsection{Session-Cookie}
Falls das im Client eingebaute Session-Handling benutzt wird, gilt Folgendes:

Wenn ein Session-Cookie vorhanden ist, und dieses Session-Cookie als "`Secure"' (nur über HTTPS abrufbar) konfiguiert ist,
geht das PiratenID-System davon aus, dass sich jemand über Session-Sicherheit Gedanken gemacht hat und übernimmt
die Konfiguration des Session-Cookies.

Ist das Secure-Flag nicht gesetzt, wird lediglich die Gültigkeitsdauer übernommen.
Der Gültigkeitsbereich des Cookies wird auf das Realm festgelegt und die Secure- und HTTP-Only-Flags werden gesetzt.

\subsection{Installation und Einrichtung}
Die Buttongrafiken (\texttt{button-*.png}) müssen in einen geeigneten öffentlichen Ordner platziert werden.
Der Pfad, unter dem die Grafiken erreicht werden können, ist im Parameter \texttt{PiratenID::\$imagepath} zu setzen (siehe unten).

Die Clientdatei (\texttt{piratenid.php}) muss irgendwo platziert werden, wo sie eingebunden werden kann,
und in den sie verwendenden PHP-Seiten eingebunden werden (beispielsweise mit \texttt{require('piratenid.php')}).
Im gleichen Verzeichnis muss die Datei certificate.pem liegen.
Diese enthält das SSL-Zertifikat, über das der PiratenID-Server authentifiziert wird.
Dabei kann es sich entweder um das Zertifikat des Servers selbst, oder um das einer CA (Zertifizierungsstelle) im Zertifikatspfad des Serverzertifikats handeln.

\subsubsection{Parameter}
Nachdem die Datei eingebunden wurde, müssen vor der Nutzung einige Parameter gesetzt werden:
\begin{itemize}
	\item \texttt{PiratenID::\$realm} - Das OpenID-Realm, den Bereich, für welchen die Authentifizierung erfolgt. Muss gesetzt werden!
										Dieser Bereich muss sich unter alleiniger Kontrolle der Webanwendung befinden.
										Diese Angabe ist sicherheitskritisch und sollte fest eingetragen werden.
										Variablen aus \texttt{\$\_SERVER} können z. T. gefälscht werden und dürfen daher hier nicht genutzt werden!
										Das Realm einer recht strengen Prüfung, es muss sich um eine HTTPS-URL auf einen Ordner handeln, genauer:
										Die URL muss mit einem Schrägstrich enden und darf keine Query-Parameter oder Anker enthalten.
										Sowohl die anfragende Seite als auch die Return-URL müssen innerhalb dieses Realms liegen.
										Das Realm wird auch für den Gültigkeitsbereich des Cookies verwendet.
										
	\item \texttt{PiratenID::\$returnurl} - Die URL, zu der der PiratenID-Server den Nutzer nach der Authentifizierung zurückschickt.
											Falls gesetzt, muss diese URL innerhalb des Realms liegen, d.h. sie muss mit der Realm-URL beginnen.
											Wird automatisch erkannt (Adresse der aktuellen Seite), falls nicht gesetzt.

	\item \texttt{PiratenID::\$imagepath} - Der öffentliche Pfad zu den Buttongrafiken. Standardmäßig leer (d.h. aktuelles Verzeichnis).
											Dieser Wert wird vor die Button-URLs eingefügt und kann ein relativer oder absoluter Pfad wie '/',
											'/piratenid-buttons/' oder 'https://images.example.com/piratenid/' sein.

	\item \texttt{PiratenID::\$attributes} - Die anzufragenden Attribute als durch Kommata getrennte Liste. Standardmäßig leer.
											Mögliche Attribute siehe Abschnitt \ref{sec:attribute}.
											Wenn hier  \texttt{mitgliedschaft-bund} aufgeführt wird, können sich auch Nichtmitglieder einloggen,
											die Webanwendung prüft die Mitgliedschaft selbst (explizite Prüfung der Mitgliedschaft).
											Wird dieses Attribut nicht abgefragt, können sich nur Mitglieder einloggen.

	\item \texttt{PiratenID::\$usePseudonym} - boolean-Wert (Standard: true), welcher angibt, ob ein Pseudonym angefragt werden soll.
											Ist dieser Wert false, erfolgt ein anonymes Login.
											Damit kann sichergestellt werden, dass sich gerade ein Pirat eingeloggt hat.
											Nutzer können jedoch nicht wiedererkannt werden, d.h. Doppelaccounts können nicht verhindert werden.
											Diese Option ist daher nur in Sonderfällen nützlich.
	
	\item \texttt{PiratenID::\$logouturl} - Die URL, auf die der Logout-Button verweisen soll. Standard: Realm-URL.
											Siehe auch \texttt{PiratenID::\$handleLogout}.

	\item \texttt{PiratenID::\$handleLogout} - Boolean, gibt an, ob das integrierte Logout-Handling verwendet werden soll.
											Wenn true, wird beim Login eine Zufallszahl erzeugt und in der Sitzung gespeichert.
											Diese wird an die Logout-URL automatisch in einem Parameter (\texttt{piratenid\_logout}) angehängt.
											\texttt{PiratenID::\$run()} erkennt diesen Parameter und loggt den Nutzer aus,
											wenn die Zufallszahl übereinstimmt. (Die Zufallszahl verhindert böswillige Logouts über CSRF.)
											Der Parameter wird automatisch aus dem \$\_GET-Array entfernt.
											Bei Verwendung von \texttt{PiratenID::\$run()} und des dadurch erzeugten Buttons
											wird das komplette Loginhandling übernommen!
											\\\textbf{Vorsicht:
												Soll statt \texttt{PiratenID::\$run()} direkt \texttt{PiratenID::\$button(...)} aufgerufen werden,
												so muss dieser Parameter deaktiviert werden! }
												
	\item \texttt{PiratenID::\$loginCallback} - Verweis auf eine Funktion\footnote{\url{http://php.net/manual/de/function.is-callable.php}},
											welche aufgerufen wird, wenn über \texttt{PiratenID::\$run()} ein Login erfolgt.
											Die Funktion erhält als Parameter das von \texttt{PiratenID::handle()} zurückgegebene Array.
											Die Funktion muss \texttt{null} zurückgeben, um das Login zu erlauben.
											Wird ein anderer Wert zurückgegeben, so wird das Login abgebrochen,
											der zurückgegebene Wert wird als Fehler angezeigt.
											Dies kann benutzt werden, um Benutzer (nach Pseudonym) zu sperren oder
											feingliedrige Filter nach anderen Attributen zu implementieren.
											Die Session ist beim Aufruf der Callback-Funktion initialisiert,
											das Ergebnis des Login-Vorgangs ist aber noch nicht darin abgelegt.
	
	\item \texttt{PiratenID::\$logoutCallback} - Verweis auf eine Funktion\footnote{\url{http://php.net/manual/de/function.is-callable.php}},
											welche aufgerufen wird, wenn über \texttt{PiratenID::\$run()} ein Logout erfolgt.
											Die Funktion erhält keine Parameter, der Rückgabewert wird ignoriert.

\end{itemize}

\subsection{Nutzung}
Für die Nutzung des Clients, nachdem die Parameter eingestellt worden sind, gibt es zwei Möglichkeiten:

\subsubsection{Nutzung über \texttt{PiratenID::run()}}
\label{sec:client-usage-basic}

Das "`Komplettpaket"' \texttt{PiratenID::run()} übernimmt sämtliche Aufgaben, die mit PiratenID verbunden sind (wie die Verarbeitung von OpenID-Antworten),
und gibt den HTML-Code für einen Login/Logout-Button zurück, der einfach an einer geeigneten Stelle ausgegeben werden kann.
Eine Session wird automatisch erzeugt, die Funktion ist daher vor allen Ausgaben aufzurufen.
Die Session-Variable \verb@$_SESSION['piratenid_user']@ wird angelegt und mit einem assoziativen Array befüllt.
Dieses Array enthält folgende Elemente:
\begin{itemize}
	\item \texttt{authenticated}: Boolean, gibt an, ob ein Nutzer eingeloggt ist.
	\item \texttt{attributes}: Assoziatives Array, enthält bei eingeloggten Nutzern die Attribute, die der Server geliefert hat.
	\item \texttt{pseudonym}: String, enthält bei pseudonym eingeloggten Nutzern ein Pseudonym, welches direkt verwendet werden kann.
								(Hash der OpenID-Identity-URL. Wenn es Angreifern gelingt, OpenID-Antworten zu fälschen,
								sie aber nicht die Identity-URL eines Nutzers kennen, können sie sich so nicht als dieser Nutzer einloggen.
								Die Identity-URL sollte aus diesem Grund nirgendwo verwendet oder angezeigt werden und wird daher nicht bereitgestellt.)
\end{itemize}

Selbstverständlich muss der zurückgegebene Button nicht genutzt werden.
Das Login-Verfahren kann in dem Fall wie im nächsten Abschnitt beschrieben eingeleitet werden.
Ein Logout erfolgt über den Aufruf der Funktion \texttt{PiratenID::logout()}.

\subsubsection{Nutzung über \texttt{PiratenID::handle()}}
\label{sec:client-usage-advanced}
Wenn mehr Kontrolle über den Authentifizierungsprozess gewünscht ist, können die Funktionen
\texttt{PiratenID::initSession()}, \texttt{PiratenID::handle()}, \texttt{PiratenID::makeOpenIDURL()} und \texttt{PiratenID::button(...)}
separat genutzt werden.

\textbf{Falls \texttt{PiratenID::button(...)} genutzt werden soll, muss der Parameter \texttt{PiratenID::\$handleLogout} deaktiviert werden!}
Geschieht dies nicht, wird in eingeloggtem Zustand dauernd die folgende Fehlermeldung angezeigt:
"`WRONG USAGE OF PIRATENID LIBRARY. Tried to generate button with \$handleLogout=true without correctly initialized session."'

\texttt{PiratenID::initSession()} initialisiert die Session auf sichere Art und Weise.
Wird ein eigenes Session Handling durchgeführt, muss diese Funktion nicht genutzt werden.

\texttt{PiratenID::makeOpenIDURL()} liefert die URL zum ID-System, die den Login-Prozess mit den derzeitigen Einstellungen einleitet.
Ein eigener Login-Button sollte auf diese URL verweisen.
Die einzelnen OpenID-Felder können auch per \texttt{PiratenID::getOpenIDRequest()} als Array abgerufen werden.

\texttt{PiratenID::handle()} verarbeitet die indirekte Antwort des PiratenID-Servers unter der Return-URL.
Bei einem POST auf die Return-URL mit dem Feld \texttt{openid\_mode} sollte diese Funktion aufgerufen werden.
Der Rückgabewert ist ein assoziatives Array mit den Feldern 
\begin{itemize}
	\item \texttt{authenticated}: Boolean, gibt an, ob ein Loginvorgang erfolgreich war
	\item \texttt{attributes}: (nur bei erfolgreichem Login) assoziatives Array mit den Attributen, die der Server geliefert hat
	\item \texttt{pseudonym}: (nur bei erfolgreichem Login mit Pseudonym) Pseudonym-String (Hash der OpenID-Identity-URL, siehe oben)
	\item \texttt{rawIdentityURL}: (nur bei erfolgreichem Login mit Pseudonym) Die OpenID-Identity-URL - sollte nicht genutzt werden, siehe oben!
	\item \texttt{error}: Enthält den aufgeretenen Fehler (oder null, wenn kein Fehler aufgetreten ist)
\end{itemize}

\texttt{PiratenID::button(\$logout, \$errortext = null)} erzeugt den HTML-Code für einen Login- bzw. Logout-Button.
Der Boolean-Parameter \$logout bestimmt, ob es sich um einen Logout-Button handelt.
Wird ein Fehlertext (\$errortext) übergehen, funktioniert der Button zwar weiter als Login/Logout-Button,
zeigt jedoch den angegebenen Fehler an.
\texttt{PiratenID::\$handleLogout} darf nicht aktiv sein, damit diese Funktion einwandfrei funktioniert.

Wenn für den Login-Status die von \texttt{PiratenID::run()} genutzte Session-Variable genutzt wird,
kann \texttt{PiratenID::logout()} zum Ausloggen des Nutzers verwendet werden und
\texttt{PiratenID::autoButton(\$errortext)} liefert automatisch den richtigen Buttontyp
(Session wird initialisiert, falls noch nicht geschehen).


\newpage
\section{Verwendung über eine OpenID-Library}
PiratenID ist kompatibel mit OpenID 2.0 mit Attribute Exchange 1.0. Ältere OpenID-Versionen wird nicht unterstützt.
Sämtliche OpenID-Transaktionen finden ausschließlich über HTTPS statt.
Der Endpoint ist \url{https://id.piratenpartei.de/openid/endpoint.php}, die XRDS-URL ist \url{https://id.piratenpartei.de/openid/xrds.php}.
Weitere Details finden sich auch in der technischen Dokumentation im Abschnitt Authentifizierung.

Bei der Verwendung mittels einer OpenID-Library ist darauf zu achten, dass die Library das SSL-Zertifikat des Servers ordnungsgemäß prüft.
Viele Libraries verzichten hier auf eine ordnungsgemäße Prüfung (z. B. werden Servername oder Herausgeber nicht geprüft), wodurch sie unsicher sind!
Nach Möglichkeit sollte die Library ausdrücklich angewiesen werden, welche Zertifikate akzeptiert werden sollen.
Dazu kann das Zertifikat verwendet werden, was dem PHP-Client beiliegt (Datei certificate.pem).

Weiterhin muss sichergestellt werden, dass OpenID-Logins nur über PiratenID, nicht aber über andere OpenID-Server, akzeptiert werden.
Es ist nicht ausreichend, die Anfrage-URL im OpenID-Request fest einzuprogrammieren,
da OpenID-Libraries oft auch Antworten verarbeiten, die sie gar nicht angefordert haben.

Bei der OpenID-Anfrage unterliegt das Realm einer recht strengen Prüfung, es muss sich um eine HTTPS-URL auf einen Ordner handeln, genauer:
Die URL muss mit einem Schrägstrich enden und darf keine Query-Parameter oder Anker enthalten.
Sowohl die anfragende Seite als auch die Return-URL müssen innerhalb dieses Realms liegen.
Das Realm wird zur Berechnung des Pseudonyms verwendet und dem Nutzer als der Name der Seite, bei der er sich anmeldet, angezeigt.

Für reguläre Authentifikation mit Pseudonym wird der identifier\_select-Modus verwendet.
Standardkonforme OpenID 2.0-Libraries sollten diesen automatisch wählen, wenn als OpenID-Identitäts-URL \url{https://id.piratenpartei.de/} angegeben wird.
Sollte dies nicht funktionieren, klappt es unter Umständen, wenn die Endpoint- oder XRDS-URL angegeben werden.
Bei identifier\_select haben die OpenID-Felder \texttt{claimed\_id} und \texttt{identity} den Wert \texttt{http://specs.openid.net/auth/2.0/identifier\_select}.

Bei einer Authentifikation mit Pseudonym liefert das ID-System eine Identity-URL der Form
\verb@https://id.piratenpartei.de/openid/pseudonym.php?id=4e2b9b5163[...]faced7@\\
wobei der Teil nach "`id="' das Pseudonym ist. Vor der weiteren Verwendung sollte das Pseudonym nochmals gehasht werden.

Für eine anonyme Authentifikation (ohne Pseudonym) dürfen im OpenID-Request \texttt{claimed\_id} und \texttt{identity} nicht mitgeschickt werden.
Nicht alle Libraries unterstützen dies.

Für die Abfrage von Attributen wird Attribute Exchange verwendet.
Die Attribute müssen mit ihren Namen (Liste siehe oben) als Alias abgefragt werden,
als Attributtyp ist jeweils "`https://id.piratenpartei.de/openid/schema/NAME"' anzugeben, wobei NAME durch den Namen des Attributs zu ersetzen ist.

Wird mit einer gewöhnlichen OpenID-Library ohne besondere Optionen eine Authentifizierung durchgeführt,
so findet eine \textbf{pseudonyme} Anmeldung ohne Abfrage weiterer Attribute statt.
Die Mitgliedschaft wird dabei \textbf{implizit} geprüft - nur Piraten können sich einloggen.
(Achtung, das bedeutet, dass ein Pirat sich nach seinem Austritt nicht mehr einloggen kann!)
Anwendungen, die OpenID bereits unterstützen, können so in der Regel mit minimalen Veränderungen (Beschränkung auf den PiratenID-Server, Zertifikate) direkt verwendet werden.


\newpage
\section{Sicherheit}

TODO verweis auf guidelines?

\end{document}