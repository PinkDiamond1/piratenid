Der bereitgestellte PiratenID-Client ist eine vollständige Eigenentwicklung.
Er ist in PHP geschrieben und ermöglicht Webanwendungen in der gleichen Sprache eine bequeme und sichere Nutzung des PiratenID-Systems.

Die Parameter des PiratenID-Systems sind fest in den Client eingebaut.
Der Client ist nicht für die Nutzung mit anderen OpenID-Servern gedacht oder geeignet.
(Es fehlen wichtige Teile des OpenID-Protokolls, welche für die Nutzung mit einem fest vorgegebenen Server jedoch nicht nötig sind.)

Der Client besteht aus einer PHP-Klasse mit statischen Funktionen und einigen statischen, öffentlichen Variablen.
Um ihn zu nutzen, werden (nach dem Einbinden der Datei) zunächst über die Variablen die gewünschten Parameter eingestellt.
Danach können die enthaltenen Funktionen benutzt werden.

Der Client kann das komplette Session-Handling übernehmen.
Hierzu müssen die betroffenen Funktionen (run bzw. initSession) aufgerufen werden, bevor die Header gesendet worden sind, d.h. vor allen Ausgaben.
Vorsicht: Ein Byte-order-mark, was manche Editoren am Anfang von UTF-8-Textdateien einfügen, kann in einer PHP-Datei dazu führen, dass die Header gesendet werden!
Wenn der Client das Session-Handling übernimmt, werden automatisch einige zusätzliche Sicherheitsmaßnahmen angewendet.
Falls es also möglich ist, sollte das Session-Handling immer dem Client überlassen werden.

Den einfachsten Einstieg in die Nutzung des Clients bietet das mitgelieferte Beispiel (\texttt{example.php}).

\subsection{Voraussetzungen}
Der Client benötigt PHP mit aktivierten Stream-Funktionen mitsamt HTTPS/SSL-Unterstützung.
Der Client wurde auf PHP 5.3.5 getestet.
Aus Sicherheitsgründen sollte stets eine aktuelle PHP-Version verwendet werden.

Die PHP-Einstellung \texttt{session.use\_only\_cookies} sollte zur Erhöhung der Sicherheit in der php.ini aktiviert werden, falls möglich.
Kann diese Einstellung nicht aktiviert werden, wird der Client die Session-ID bei jedem Aufruf neu erzeugen,
um session fixation-Angriffe zu verhindern.
Dies kann die Stabilität der Sessions beeinträchtigen.

\subsubsection{Session-Cookie}
Falls das im Client eingebaute Session-Handling benutzt wird, gilt Folgendes:

Wenn ein Session-Cookie vorhanden ist, und dieses Session-Cookie als "`Secure"' (nur über HTTPS abrufbar) konfiguiert ist,
geht das PiratenID-System davon aus, dass sich jemand über Session-Sicherheit Gedanken gemacht hat und übernimmt
die Konfiguration des Session-Cookies.

Ist das Secure-Flag nicht gesetzt, wird lediglich die Gültigkeitsdauer übernommen.
Der Gültigkeitsbereich des Cookies wird auf das Realm festgelegt und die Secure- und HTTP-Only-Flags werden gesetzt.

\subsection{Installation und Einrichtung}
Die Buttongrafiken (\texttt{button-*.png}) müssen in einen geeigneten öffentlichen Ordner platziert werden.
Der Pfad, unter dem die Grafiken erreicht werden können, ist im Parameter \texttt{PiratenID::\$imagepath} zu setzen (siehe unten).

Die Clientdatei (\texttt{piratenid.php}) muss irgendwo platziert werden, wo sie eingebunden werden kann,
und in den sie verwendenden PHP-Seiten eingebunden werden (beispielsweise mit \texttt{require('piratenid.php')}).
Im gleichen Verzeichnis muss die Datei certificate.pem liegen.
Diese enthält das SSL-Zertifikat, über das der PiratenID-Server authentifiziert wird.
Dabei kann es sich entweder um das Zertifikat des Servers selbst, oder um das einer CA (Zertifizierungsstelle) im Zertifikatspfad des Serverzertifikats handeln.

\subsubsection{Parameter}
Nachdem die Datei eingebunden wurde, müssen vor der Nutzung einige Parameter gesetzt werden:
\begin{itemize}
	\item \texttt{PiratenID::\$realm} - Das OpenID-Realm, den Bereich, für welchen die Authentifizierung erfolgt. Muss gesetzt werden!
										Dieser Bereich muss sich unter alleiniger Kontrolle der Webanwendung befinden.
										Diese Angabe ist sicherheitskritisch und sollte fest eingetragen werden.
										Variablen aus \texttt{\$\_SERVER} können z. T. gefälscht werden und dürfen daher hier nicht genutzt werden!
										Das Realm einer recht strengen Prüfung, es muss sich um eine HTTPS-URL auf einen Ordner handeln, genauer:
										Die URL muss mit einem Schrägstrich enden und darf keine Query-Parameter oder Anker enthalten.
										Sowohl die anfragende Seite als auch die Return-URL müssen innerhalb dieses Realms liegen.
										Das Realm wird auch für den Gültigkeitsbereich des Cookies verwendet.
										
	\item \texttt{PiratenID::\$returnurl} - Die URL, zu der der PiratenID-Server den Nutzer nach der Authentifizierung zurückschickt.
											Falls gesetzt, muss diese URL innerhalb des Realms liegen, d.h. sie muss mit der Realm-URL beginnen.
											Wird automatisch erkannt (Adresse der aktuellen Seite), falls nicht gesetzt.

	\item \texttt{PiratenID::\$imagepath} - Der öffentliche Pfad zu den Buttongrafiken. Standardmäßig leer (d.h. aktuelles Verzeichnis).
											Dieser Wert wird vor die Button-URLs eingefügt und kann ein relativer oder absoluter Pfad wie '/',
											'/piratenid-buttons/' oder 'https://images.example.com/piratenid/' sein.

	\item \texttt{PiratenID::\$attributes} - Die anzufragenden Attribute als durch Kommata getrennte Liste. Standardmäßig leer.
											Mögliche Attribute siehe Abschnitt \ref{sec:attribute}.
											Wenn hier  \texttt{mitgliedschaft-bund} aufgeführt wird, können sich auch Nichtmitglieder einloggen,
											die Webanwendung prüft die Mitgliedschaft selbst (explizite Prüfung der Mitgliedschaft).
											Wird dieses Attribut nicht abgefragt, können sich nur Mitglieder einloggen.

	\item \texttt{PiratenID::\$usePseudonym} - boolean-Wert (Standard: true), welcher angibt, ob ein Pseudonym angefragt werden soll.
											Ist dieser Wert false, erfolgt ein anonymes Login.
											Damit kann sichergestellt werden, dass sich gerade ein Pirat eingeloggt hat.
											Nutzer können jedoch nicht wiedererkannt werden, d.h. Doppelaccounts können nicht verhindert werden.
											Diese Option ist daher nur in Sonderfällen nützlich.
	
	\item \texttt{PiratenID::\$logouturl} - Die URL, auf die der Logout-Button verweisen soll. Standard: Realm-URL.
											Siehe auch \texttt{PiratenID::\$handleLogout}.

	\item \texttt{PiratenID::\$handleLogout} - Boolean, gibt an, ob das integrierte Logout-Handling verwendet werden soll.
											Wenn true, wird beim Login eine Zufallszahl erzeugt und in der Sitzung gespeichert.
											Diese wird an die Logout-URL automatisch in einem Parameter (\texttt{piratenid\_logout}) angehängt.
											\texttt{PiratenID::\$run()} erkennt diesen Parameter und loggt den Nutzer aus,
											wenn die Zufallszahl übereinstimmt. (Die Zufallszahl verhindert böswillige Logouts über CSRF.)
											Der Parameter wird automatisch aus dem \$\_GET-Array entfernt.
											Bei Verwendung von \texttt{PiratenID::\$run()} und des dadurch erzeugten Buttons
											wird das komplette Loginhandling übernommen!
											\\\textbf{Vorsicht:
												Soll statt \texttt{PiratenID::\$run()} direkt \texttt{PiratenID::\$button(...)} aufgerufen werden,
												so muss dieser Parameter deaktiviert werden! }
												
	\item \texttt{PiratenID::\$loginCallback} - Verweis auf eine Funktion\footnote{\url{http://php.net/manual/de/function.is-callable.php}},
											welche aufgerufen wird, wenn über \texttt{PiratenID::\$run()} ein Login erfolgt.
											Die Funktion erhält als Parameter das von \texttt{PiratenID::handle()} zurückgegebene Array.
											Die Funktion muss \texttt{null} zurückgeben, um das Login zu erlauben.
											Wird ein anderer Wert zurückgegeben, so wird das Login abgebrochen,
											der zurückgegebene Wert wird als Fehler angezeigt.
											Dies kann benutzt werden, um Benutzer (nach Pseudonym) zu sperren oder
											feingliedrige Filter nach anderen Attributen zu implementieren.
											Die Session ist beim Aufruf der Callback-Funktion initialisiert,
											das Ergebnis des Login-Vorgangs ist aber noch nicht darin abgelegt.
	
	\item \texttt{PiratenID::\$logoutCallback} - Verweis auf eine Funktion\footnote{\url{http://php.net/manual/de/function.is-callable.php}},
											welche aufgerufen wird, wenn über \texttt{PiratenID::\$run()} ein Logout erfolgt.
											Die Funktion erhält keine Parameter, der Rückgabewert wird ignoriert.

\end{itemize}

\subsection{Nutzung}
Für die Nutzung des Clients, nachdem die Parameter eingestellt worden sind, gibt es zwei Möglichkeiten:

\subsubsection{Nutzung über \texttt{PiratenID::run()}}
\label{sec:client-usage-basic}

Das "`Komplettpaket"' \texttt{PiratenID::run()} übernimmt sämtliche Aufgaben, die mit PiratenID verbunden sind (wie die Verarbeitung von OpenID-Antworten),
und gibt den HTML-Code für einen Login/Logout-Button zurück, der einfach an einer geeigneten Stelle ausgegeben werden kann.
Eine Session wird automatisch erzeugt, die Funktion ist daher vor allen Ausgaben aufzurufen.
Die Session-Variable \verb@$_SESSION['piratenid_user']@ wird angelegt und mit einem assoziativen Array befüllt.
Dieses Array enthält folgende Elemente:
\begin{itemize}
	\item \texttt{authenticated}: Boolean, gibt an, ob ein Nutzer eingeloggt ist.
	\item \texttt{attributes}: Assoziatives Array, enthält bei eingeloggten Nutzern die Attribute, die der Server geliefert hat.
	\item \texttt{pseudonym}: String, enthält bei pseudonym eingeloggten Nutzern ein Pseudonym, welches direkt verwendet werden kann.
								(Hash der OpenID-Identity-URL. Wenn es Angreifern gelingt, OpenID-Antworten zu fälschen,
								sie aber nicht die Identity-URL eines Nutzers kennen, können sie sich so nicht als dieser Nutzer einloggen.
								Die Identity-URL sollte aus diesem Grund nirgendwo verwendet oder angezeigt werden und wird daher nicht bereitgestellt.)
\end{itemize}

Selbstverständlich muss der zurückgegebene Button nicht genutzt werden.
Das Login-Verfahren kann in dem Fall wie im nächsten Abschnitt beschrieben eingeleitet werden.
Ein Logout erfolgt über den Aufruf der Funktion \texttt{PiratenID::logout()}.

\subsubsection{Nutzung über \texttt{PiratenID::handle()}}
\label{sec:client-usage-advanced}
Wenn mehr Kontrolle über den Authentifizierungsprozess gewünscht ist, können die Funktionen
\texttt{PiratenID::initSession()}, \texttt{PiratenID::handle()}, \texttt{PiratenID::makeOpenIDURL()} und \texttt{PiratenID::button(...)}
separat genutzt werden.

\textbf{Falls \texttt{PiratenID::button(...)} genutzt werden soll, muss der Parameter \texttt{PiratenID::\$handleLogout} deaktiviert werden!}
Geschieht dies nicht, wird in eingeloggtem Zustand dauernd die folgende Fehlermeldung angezeigt:
"`WRONG USAGE OF PIRATENID LIBRARY. Tried to generate button with \$handleLogout\=true without correctly initialized session."'

\texttt{PiratenID::initSession()} initialisiert die Session auf sichere Art und Weise.
Wird ein eigenes Session Handling durchgeführt, muss diese Funktion nicht genutzt werden.

\texttt{PiratenID::makeOpenIDURL()} liefert die URL zum ID-System, die den Login-Prozess mit den derzeitigen Einstellungen einleitet.
Ein eigener Login-Button sollte auf diese URL verweisen.
Die einzelnen OpenID-Felder können auch per \texttt{PiratenID::getOpenIDRequest()} als Array abgerufen werden.

\texttt{PiratenID::handle()} verarbeitet die indirekte Antwort des PiratenID-Servers unter der Return-URL.
Bei einem POST auf die Return-URL mit dem Feld \texttt{openid\_mode} sollte diese Funktion aufgerufen werden.
Der Rückgabewert ist ein assoziatives Array mit den Feldern 
\begin{itemize}
	\item \texttt{authenticated}: Boolean, gibt an, ob ein Loginvorgang erfolgreich war
	\item \texttt{attributes}: (nur bei erfolgreichem Login) assoziatives Array mit den Attributen, die der Server geliefert hat
	\item \texttt{pseudonym}: (nur bei erfolgreichem Login mit Pseudonym) Pseudonym-String (Hash der OpenID-Identity-URL, siehe oben)
	\item \texttt{rawIdentityURL}: (nur bei erfolgreichem Login mit Pseudonym) Die OpenID-Identity-URL - sollte nicht genutzt werden, siehe oben!
	\item \texttt{error}: Enthält den aufgeretenen Fehler (oder null, wenn kein Fehler aufgetreten ist)
\end{itemize}

\texttt{PiratenID::button(\$logout, \$errortext = null)} erzeugt den HTML-Code für einen Login- bzw. Logout-Button.
Der Boolean-Parameter \$logout bestimmt, ob es sich um einen Logout-Button handelt.
Wird ein Fehlertext (\$errortext) übergehen, funktioniert der Button zwar weiter als Login/Logout-Button,
zeigt jedoch den angegebenen Fehler an.
\texttt{PiratenID::\$handleLogout} darf nicht aktiv sein, damit diese Funktion einwandfrei funktioniert.

Wenn für den Login-Status die von \texttt{PiratenID::run()} genutzte Session-Variable genutzt wird,
kann \texttt{PiratenID::logout()} zum Ausloggen des Nutzers verwendet werden und
\texttt{PiratenID::autoButton(\$errortext)} liefert automatisch den richtigen Buttontyp
(Session wird initialisiert, falls noch nicht geschehen).
