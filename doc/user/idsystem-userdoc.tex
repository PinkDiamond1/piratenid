\documentclass[parskip=half]{scrartcl}

\usepackage[utf8]{inputenc}
\usepackage[ngerman]{babel}
\usepackage{amsmath}
\usepackage{amsfonts}
\usepackage{amssymb}
\usepackage{color}
\usepackage{hyperref}
\usepackage{graphicx}
\usepackage{epstopdf}
\usepackage{listings}
\usepackage{fancyvrb}
\definecolor{lnkcol}{rgb}{0.0,0.0,0.6}
\hypersetup{
	pdfstartview=FitH,
	colorlinks=true,
	urlcolor=lnkcol,
	citecolor=lnkcol,
	linkcolor=lnkcol,
	filecolor=lnkcol
}

\title{PiratenID: Handbuch für Benutzer}
\author{Jan Schejbal}
\date{}

\bibliographystyle{unsrt}

\setcounter{secnumdepth}{2}

\begin{document}
\maketitle

\begin{abstract}Dieses Dokument enthält eine Anleitung für Nutzer des PiratenID-Systems.
Für Entwickler von Webanwendungen existiert eine separate Anleitung\footnote{TODO}.
Technische Details zum ID-System finden sich in der technischen Dokumentation\footnote{TODO}.
\end{abstract}


\section{Einführung}
Das PiratenID-System erlaubt es Piraten, ihren Mitgliedschaftsstatus gegenüber von Dritten betriebenen Webseiten nachzuweisen,
ohne dass diese Websites Zugriff auf die Mitgliederdatenbank bekommen und ohne mehr Daten als nötig offenzulegen.

Das PiratenID-System ist unter \url{https://id.piratenpartei.de/} zu erreichen (das "`https"' muss mit eingegeben werden!).
Bei einem Login über das ID-System leitet eine Website dich als den Nutzer auf das ID-System um, du loggst dich dort ein,
und das ID-System bestätigt das Login und leitet die von dir freigegebenen Informationen an die Website weiter.

\subsection{Teilnahme}
Die Teilnahme am ID-System ist freiwillig.

Bei der Accounterstellung wählst du einen Nutzernamen, ein Kennwort, und gibst deine E-Mail-Adresse ein.
Jeder Pirat erhält ein Token - mit diesem kannst du deinen Account voll aktivieren.
Über die Eingabe des Tokens wird der Account pseudonym mit den Gliederungen verknüpft, in denen du Mitglied bist.
Das ID-System erfährt hierbei weder deinen Namen noch die Mitgliedsnummer!
Die Zuordnung zwischen Token und Mitgliedern ist nur in der Mitgliederdatenbank der Piratenpartei gespeichert.

\subsection{Nutzung}
Webseiten können dich dazu auffordern, dass du dich über das ID-System einloggst, und dabei bestimmte Informationen über dich abfragen.
Du wirst dabei auf die Seite des ID-Systems umgeleitet und bekommst dort angezeigt, welche Seite anfragt und welche Informationen übermittelt werden sollen.
Wenn du mit der Übermittlung einverstanden bist, bestätigst du dies, indem du deinen Benutzernamen und dein Kennwort eingibst.
Du wirst daraufhin wieder zurück zur Website geleitet und solltest dort nun eingeloggt sein.

Webseiten können folgende Informationen über dich anfordern:
\begin{itemize}
	\item Mitgliedschaft in der Piratenpartei - diese wird immer abgefragt, denn der Zweck des Systems ist, dir die Möglichkeit zu geben, dich als Pirat "`auszuweisen"'.
		Entweder diese wird implizit abgefragt, dann kannst du dich nur einloggen, wenn du Pirat bist und ein Token eingetragen hast,
		oder explizit, dann können sich auch Nichtpiraten einloggen, die Website bekommt aber mitgeteilt, ob du Pirat bist.
	
	\item Mitgliedschaften in weiteren Gliederungen (LV, BZV, KV, OV) - diese können optional angefordert werden.
		Sind für eine Gliederungsstufe keine Informationen vorhanden (Anfangs werden voraussichtlich nur LVs angegeben),
		oder bist du in keinem Verband der entsprechenden Stufe Mitglied,
		wird ein leeres oder als "`keine Mitgliedschaft"' gekennzeichnetes Feld übermittelt.
		
	\item Pseudonym - ein zufälliges Pseudonym, zu dem im folgenden Abschnitt mehr erläutert wird
\end{itemize}

\subsection{Das Pseudonym}
Damit eine Website dich wiedererkennen kann, z. B. damit du in einer Umfrage nur einmal abstimmen kannst, kann sie ein Pseudonym abfragen.
Dies ist ein zufälliger Wert, der für jede Website und jeden Nutzer anders ist.
Wenn du dich also bei zwei Websites über das ID-System einloggst, hast du für jede Website ein eigenes Pseudonym.
Die Websites haben auch keine Möglichkeit, diese Pseudonyme miteinander zu verknüpfen.
Website A kann also nicht feststellen, ob ein Nutzer X der gleiche Pirat ist wie Nutzer Y auf Website B - auch nicht, wenn die beiden Websites zusammenarbeiten.

Solange das entsprechende Benutzerkonto im ID-System existiert, wäre es den Betreibern des ID-Systems mit einem gewissen Aufwand möglich, ein Pseudonym einem Nutzer zuzuordnen.
Ohne Zugriff auf die Datenbank des ID-Systems sowie einen weiteren geheimen Wert ist eine solche Zuordnung nicht möglich.

\newpage
\section{Sicherheit}
Um das ID-System sicher nutzen zu können, musst auch du auf einige Dinge achten.
Viele dieser Ratschläge gelten nicht speziell für das ID-System, sondern allgemein für die sichere Nutzung von PC und Internet.

\subsection{Login nur beim ID-System}
Dein Kennwort darfst du nur auf der Seite des ID-Systems eingeben.
Nachdem du von einer Website zur Login-Seite weitergeleitet wurdest, mussst du also vor der Eingabe des Kennworts \textbf{immer} prüfen,
ob du dich tatsächlich auf der echten Seite befindest.
Die Adresse muss mit \url{https://id.piratenpartei.de/} beginnen. Tut sie dies nicht, darfst du dein Kennwort nicht eingeben!
(Melde solche Vorfälle bitte immer der BundesIT!)

Die Verbindung zum ID-System erfolgt immer verschlüsselt über das HTTPS-Protokoll, damit das Kennwort nicht abgehört werden kann.
Dies wird je nach verwendetem Browser durch entsprechende Symbole gekennzeichnet, meist zusammen mit dem Servernamen (id.piratenpartei.de).
Du solltest lernen, wie dein Browser das anzeigt!

Google Chrome zeigt ein grünes Schlosssymbol neben der Adresse und hebt "`https"' grün und den Servernamen schwarz hervor.
Mozilla Firefox 4 zeigt die Domain (piratenpartei.de) in einem blauen Kasten neben der Adresse.
Alternativ kann (bei beiden Browsern) der Name des Seiteninhabers (Piratenpartei Deutschland) auf grünem Grund neben der Adresse angezeigt werden.
(Dies passiert, wenn die Verbindung mit einer besonders sicheren Art von Zertifikat abgesichert wird.)

\subsection{Sicheres Kennwort}
Das Kennwort schützt dein Benutzerkonto vor unbefugtem Zugriff.
Du solltest es daher nie anderen Personen mitteilen - das bedeutet auch, dass du das gleiche Kennwort nicht bei mehreren Webseiten nutzen solltest!
Ein sicheres Kennwort kannst du beispielsweise bilden, indem du 4 Wörter aneinanderreihst:
"`\texttt{lastwagen elefant geldbörse buch}"' wäre beispielsweise ein sehr sicheres Kennwort.

Wenn du dir dein Kennwort irgendwo aufschreibst, musst du den Zettel sicher verwahren!
Sollten andere Personen dein Kennwort erfahren haben, ändere es, sobald du kannst.

\subsection{Sicherer Client}
Du solltest das ID-System nur von Geräten (PCs, Smartphones, ...) aus nutzen, die sicher und vertrauenswürdig sind.

Die wichtigste Maßnahme zum Absichern eines Geräts ist, sämtliche enthaltene Software aktuell zu halten.
Dazu gehört zunächst das Betriebssystem.
Unter Windows musst du die Windows-Updates über das eingebaute Updatesystem durchführen. Andere Betriebssysteme haben ähnliche Möglichkeiten.

Auch dein Browser muss regelmäßig aktualisiert werden, um bekanntgewordene Sicherheitslücken zu schließen.
Die meisten aktuellen Browser enthalten eine eingebaute Updatefunktion.
Den Internet Explorer solltest du nicht benutzen, da Sicherheitslücken oft erst mit großer Verzögerung geschlossen werden.
Empfohlene Browser sind z. B. Mozilla Firefox\footnote{\url{https://www.mozilla.com/de/firefox/}}
oder Google Chrome\footnote{\url{http://www.google.com/chrome/intl/de/landing_tv.html}}.
Veraltete Versionen von Internet Explorer werden vom ID-System nicht unterstützt.

Für bestimmte Dateiformate werden oft Plug-Ins installiert.
Die bekanntesten sind der Adobe Reader(für das Öffnen von PDF-Dateien),
Flash (für das Abspielen von Animationen) sowie
Java (für in Webseiten eingebettete Programme).
Diese Plugins sind zugleich beliebte Einfallstore für Schadsoftware (Viren).
Daher musst du auch diese Plugins immer aktuell halten, wenn sie installiert sind.
Verwende dazu am Besten die eingebaute Updatefunktion.

Ein aktuell gehaltenes Virenschutzprogramm kann die Sicherheit des Systems weiter erhöhen,
bietet jedoch keinen 100\%-igen Schutz und kann das Aktualisieren des Systems auf keinen Fall ersetzen!

Software sollte nur aus vertrauenswürdigen Quellen bezogen und installiert werden.
Neuere Windows-Versionen zeigen beim Starten von aus dem Internet heruntergeladenen Dateien an, von welchem Herausgeber diese stammen.
Achte auf diese Information. Bei Software von großen Firmen steht fast nie "`Herausgeber: Unbekannt"' in diesem Dialogfeld.

\subsection{Sichere E-Mail-Adresse}
Wenn du dein Kennwort vergisst, kannst du dir einen Rücksetzlink per E-Mail schicken lassen.
Darüber kannst du dann ein neues Kennwort setzen.
Wenn also jemand Unbefugtes Zugriff auf dein E-Mail-Postfach bekommt,
kann er das Kennwort zurücksetzen und sich so Zugang verschaffen!
Das gilt nicht nur für das ID-System, sondern auch für zahlreiche andere Dienste.
Die hier genannten Sicherheitsmaßnahmen (sicheres Kennwort, sicherer Computer, ...) solltest du also auch für den Zugang zum E-Mail-Postfach einhalten!

\newpage
\section{Anleitung}
Dieser Abschnitt erklärt Schritt für Schritt, wie du das ID-System nutzen kannst.

\subsection{Account erstellen}
Um einen neuen Account (Benutzerkonto) zu erstellen, besuche die Website des ID-Systems \url{https://id.piratenpartei.de/} und klicke auf "`Account erstellen"'.
Fülle das Formular aus.
Benutzername und Mailadresse können später nicht mehr geändert werden.
Gib unbedingt eine Mailadresse ein, die ausschließlich dir gehört, denn wer Zugriff auf das entsprechende Postfach hat, kann das Kennwort des Accounts zurücksetzen.
Du darfst gerne einen Benutzernamen und eine Mailadresse verwenden, die keine Rückschlüsse auf deine Person zulassen.

Sobald du das Formular abgeschickt hast, solltest du eine E-Mail mit einem Aktivierungslink erhalten.
Öffne diesen, damit dein Account aktiviert wird.
Solltest du keine Mail erhalten, schaue auch im Spamordner nach.
Wenn dein Account nicht innerhalb einer bestimmten Zeit aktiviert wird, wird er wieder gelöscht.

Sobald du deinen Account aktiviert hast, kannst du die weiteren Schritte durchführen.
Als nächstes solltest du dein Token eintragen, damit das System dich als Mitglied erkennt.

\subsection{Token eintragen}
Jedes Mitglied erhält ein Token. Darüber kannst du im ID-System deine Piraten- und Gliederungsmitgliedschaften eintragen.
\textbf{Beachte:} Sobald du dein Token eingetragen hast, ist es für immer an deinen Account gebunden.
Du kannst das Token in deinem Account nicht mehr ändern oder löschen, und du kannst es auch nicht für einen anderen Account verwenden.
Wenn du also z. B. einen Tippfehler in deinem Benutzernamen entdeckt hast, solltest du das Token nicht eintragen.
Auch wenn du deinen Acount löschst, wird dein Token \textbf{nicht} wieder freigegeben (siehe unter "`Account löschen"').

Um ein Token einzutragen, wähle auf der Seite des ID-Systems "`Token eingeben"', fülle das Formular aus und sende es ab.

\subsection{Kennwort ändern}
Über den entsprechenden Menüpunkt kannst du dein Kennwort ändern.
Es gelten die gleichen Sicherheitsanforderungen wie beim Erstellen des Benutzerkontos (Mindestens 8 Zeichen, mindestens 2 Arten von Zeichen).

\subsection{Benutzername oder Kennwort vergessen}
Wenn du deinen Benutzernamen oder dein Kennwort vergessen hast, klicke auf "`Login vergessen"' und gib deine E-Mail-Adresse ein.
Du erhältst eine Mail mit deinem Benutzernamen sowie einem Link.
Wenn du ein neues Kennwort setzen möchtest, klicke auf den Link und du erhältst die Möglichkeit, ein neues Kennwort für dein Benutzerkonto zu vergeben.

\subsection{Einloggen}
Um dich über PiratenID auf einer Website einzuloggen, musst du den Login-Vorgang auf der jeweiligen Website starten.
Du solltest dann auf die Website des ID-Systems umgeleitet werden.
Prüfe, dass du dich tatsächlich auf der richtigen Seite befindest (siehe unter "`Sicherheit'"),
welche Daten abgefragt werden und welche Seite diese haben möchte,
und gib deinen Benutzernamen und dein Kennwort ein, wenn du mit der Weitergabe einverstanden bist.
Du wirst dann auf die Seite zurückgeleitet und solltest eingeloggt sein.

Wenn du dich ausloggen möchtest, musst du das über die Website machen, die du verwendest.

\subsection{Account löschen}
Wenn du deinen Account löschen möchtest, kannst du dies über den entsprechenden Menüpunkt tun.
Lies sorgfältig den entsprechenden Warnhinweis!

Wenn du deinen Account löschst, verlierst du sämtliche Pseudonyme und kannst dich mit diesen somit nicht mehr einloggen.
Falls in deinem Account bereits ein Token eingetragen ist, bleibt dieses als verbraucht markiert.
Dir wird \textbf{kein} neues Token ausgestellt.
\textbf{Wenn du einen Account mit eingetragenem Token löschst, verlierst du dauerhaft die Möglichkeit, das ID-System als Pirat zu nutzen.}

Solltest du deinen Account mit eingetragenem Token versehentlich gelöscht haben, kontaktiere \textbf{umgehend} die BundesIT.
Wenn du Glück hast, existiert noch ein Backup, aus dem dein Benutzerkonto wiederhergestellt werden kann.

\newpage
\section{Sonstiges}
\subsection{Token verloren}
Solltest du dein Token verloren haben, muss zwischen zwei Fällen unterschieden werden:

Wenn du dein Token noch nicht benutzt hast, kannst du dir ein neues Token ausstellen lassen.
Wende dich hierzu bitte an deinen Landes-GenSek. Dieser muss dein altes Token sperren,
und bei der IT die Bestätigung einholen, dass es noch nicht verwendet wurde.
Anschließend kann dir ein neues Token ausgestellt werden.

Wenn du dein Token bereits benutzt hast, benötigst du es eigentlich nicht mehr.
Es kann dir auch kein neues Token ausgestellt werden.
Sofern du die Zugangsdaten für deinen Account noch kennst, kannst du ihn ganz normal benutzen.
Hast du deinen Benutzernamen und/oder dein Kennwort vergessen, gehe wie unter "`Benutzername oder Kennwort vergessen"' beschrieben vor.
Falls du auch so keinen Zugriff mehr auf deinen Account bekommst,
z. B. weil du nicht mehr weißt, welche E-Mail-Adresse du verwendet hast oder du nicht mehr auf dein Postfach zugreifen kannst,
kann dein Landes-GenSek der IT auf deinen Wunsch hin den Hash des Tokens mitteilen.
Darüber kann die IT deinen Account dir zuordnen und z. B. eine neue Adresse setzen.

Beide Szenarien sind mit viel Arbeit für dich, deinen GenSek und die IT verbunden.
Versuche also bitte, dein Token und deinen Account nicht verlieren.

\subsection{Sicherheitslücken, Angriffe}
Sollten dir Sicherheitslücken im ID-System auffallen oder solltest du Angriffsversuche (z. B. Phishing) auf das ID-System beobachten,
kontaktiere bitte die IT auf dem unten angegebenen Weg.

\subsection{Kontakt}
Bei weiteren Fragen kontaktiere bitte die BundesIT der Piratenpartei.
Wie das geht, wird unter \url{http://wiki.piratenpartei.de/IT/Kontakt} beschrieben.

TODO Mailqueue für PiratenID?

\end{document}